\documentclass[10pt, letterpaper]{article}

% Packages:
\usepackage[
    ignoreheadfoot,
    top=2 cm,
    bottom=2 cm,
    left=2 cm,
    right=2 cm,
    footskip=1.0 cm,
]{geometry}
\usepackage{titlesec}
\usepackage{tabularx}
\usepackage{array}
\usepackage[dvipsnames]{xcolor} 
\definecolor{primaryColor}{RGB}{0, 0, 0}
\usepackage{enumitem}
\usepackage{fontawesome5}
\usepackage{amsmath}
\usepackage[
    pdftitle={Nuthan Reddy's Resume},
    pdfauthor={Nuthan Reddy},
    colorlinks=true,
    urlcolor=primaryColor
]{hyperref}
\usepackage[pscoord]{eso-pic}
\usepackage{calc}
\usepackage{bookmark}
\usepackage{lastpage}
\usepackage{changepage}
\usepackage{paracol}
\usepackage{ifthen}
\usepackage{needspace}
\usepackage{iftex}

\ifPDFTeX
    \input{glyphtounicode}
    \pdfgentounicode=1
    \usepackage[T1]{fontenc}
    \usepackage[utf8]{inputenc}
    \usepackage{lmodern}
\fi

% Use a font with proper bold support
\usepackage[T1]{fontenc}
\usepackage{charter}

% Ensure bold font works
\usepackage[T1]{fontenc}
% Font configuration - Switch to Helvetica (Arial-like) for better Tectonic compatibility
\usepackage{tgheros} 
\renewcommand{\familydefault}{\sfdefault}

% Settings:
\raggedright
\AtBeginEnvironment{adjustwidth}{\partopsep0pt}
\pagestyle{empty}
\setcounter{secnumdepth}{0}
\setlength{\parindent}{0pt}
\setlength{\topskip}{0pt}
\setlength{\columnsep}{0.15cm}
\pagenumbering{gobble}

\titleformat{\section}{\needspace{4\baselineskip}\bfseries\large}{}{0pt}{}[\vspace{1pt}\titlerule]
\titlespacing{\section}{-1pt}{0.3 cm}{0.2 cm}

\renewcommand\labelitemi{$\vcenter{\hbox{\small$\bullet$}}$}

\newenvironment{highlights}{
    \begin{itemize}[
        topsep=0.10 cm,
        parsep=0.10 cm,
        partopsep=0pt,
        itemsep=0pt,
        leftmargin=0 cm + 10pt
    ]
}{
    \end{itemize}
}

\newenvironment{onecolentry}{
    \begin{adjustwidth}{0 cm + 0.00001 cm}{0 cm + 0.00001 cm}
}{
    \end{adjustwidth}
}

\newenvironment{twocolentry}[2][]{
    \onecolentry
    \def\secondColumn{#2}
    \setcolumnwidth{\fill, 4.5 cm}
    \begin{paracol}{2}
}{
    \switchcolumn \raggedleft \secondColumn
    \end{paracol}
    \endonecolentry
}

\newenvironment{header}{
    \setlength{\topsep}{0pt}\par\kern\topsep\centering\linespread{1.5}
}{
    \par\kern\topsep
}

\begin{document}

\begin{header}
    \fontsize{13 pt}{13 pt}\selectfont \textbf{NUTHAN REDDY VADDI REDDY}
    
    \vspace{0.5pt}
    
    \normalsize
    nuthanreddy001@gmail.com\hspace{0.5pt}| \hspace{0.5pt}
    +1682-406-56-46\hspace{0.5pt}| \hspace{0.5pt}
    github.com/Nuthanreddy05\hspace{0.5pt}| \hspace{0.5pt}
    www.linkedin.com/in/nuthan-reddy-vaddi-reddy
\end{header}

\section{Summary}
\begin{onecolentry}
    Software Engineer with 3+ years of experience in cloud-native systems and rapid prototyping for innovation initiatives. Specialized in Python, Java, AWS, and Infrastructure as Code (IaC). Improved system reliability by 24\% and optimized API latency by 31\% in enterprise environments. Currently developing scalable cloud foundation services and exploring emerging technologies at a leading financial institution.
\end{onecolentry}

\vspace{0.2cm}

\section{Experience}

\begin{twocolentry}{May 2024 -- Present}
    \textbf{Software Engineer}, Albertsons -- Dallas, TX
\end{twocolentry}
\begin{onecolentry}
\begin{highlights}
    
    \item Built and maintained cloud-based systems that helped the company's innovation teams rapidly test new technologies and develop scalable solutions for financial services
    
    \item Reduced AWS infrastructure costs by \$45K annually by implementing Infrastructure as Code using Terraform and CloudFormation, eliminating manual provisioning and improving deployment consistency by 99\%
    
    \item Improved system reliability by 24\% by designing and deploying microservices using Python, Java, and Docker containers on AWS EKS, enabling independent scaling of critical components
    
    \item Accelerated prototype development by 3x by building rapid prototyping frameworks using Python and serverless technologies (Lambda), reducing time-to-market for innovation projects
    
    \item Enhanced security posture by 28\% by implementing identity and access management solutions using AWS IAM and security best practices across cloud-native applications
    
    \item Optimized CI/CD pipelines using Jenkins and GitHub Actions, reducing deployment time by 35\% and enabling daily releases for innovation initiatives
    
    \item Integrated cloud monitoring tools (CloudWatch, Prometheus) across distributed systems, improving incident detection time by 40\% and reducing mean-time-to-resolution by 25\%
    
    \item Designed and implemented scalable database solutions using both RDBMS (PostgreSQL) and NoSQL (DynamoDB), improving query performance by 32\% for high-traffic applications
    
\end{highlights}
\end{onecolentry}

\vspace{0.2cm}

\begin{twocolentry}{May 2020 -- July 2023}
    \textbf{Software Engineer}, ValueLabs -- Hyderabad
\end{twocolentry}
\begin{onecolentry}
\begin{highlights}
    
    \item Supported enterprise clients by building and maintaining web applications that automated business processes and improved operational efficiency across multiple industries
    
    \item Developed microservices-based applications using Java and Python, processing 500GB of transactional data daily with 99.5\% reliability for financial and retail clients
    
    \item Reduced deployment time by 60\% by implementing containerized solutions using Docker and Kubernetes, enabling faster feature delivery for enterprise applications
    
    \item Improved application performance by 26\% by optimizing SQL queries and database schemas across PostgreSQL and MySQL systems handling complex business logic
    
    \item Implemented automated testing frameworks using Python and JUnit, reducing regression defects by 22\% and improving code quality across development teams
    
\end{highlights}
\end{onecolentry}

\vspace{0.2cm}


\section{Academic Project Experience}


\begin{twocolentry}{Jan 2024 – May 2024}
    \textbf{Cloud-Native Innovation Platform (Academic Capstone)}
\end{twocolentry}
\begin{onecolentry}
\begin{highlights}
    
    \item Designed and deployed a scalable cloud-native platform using AWS EKS, Docker, and Terraform, supporting 10K+ concurrent users in load testing scenarios
    
    \item Implemented microservices architecture using Python and Java, with REST APIs achieving 99.9\% availability and sub-100ms response times under peak load
    
    \item Built comprehensive CI/CD pipelines using Jenkins and GitHub Actions, enabling automated testing and deployment with zero-downtime updates
    
    \item Integrated cloud monitoring and logging using Prometheus, Grafana, and CloudWatch, providing real-time visibility into system performance and health metrics
    
\end{highlights}
\end{onecolentry}

\vspace{0.2cm}

\begin{twocolentry}{Aug 2023 – Dec 2023}
    \textbf{Infrastructure as Code Automation Framework (Personal Project)}
\end{twocolentry}
\begin{onecolentry}
\begin{highlights}
    
    \item Developed reusable Terraform modules for AWS infrastructure provisioning, reducing deployment time by 75\% for common cloud patterns
    
    \item Implemented multi-cloud compatibility using Terraform and Azure Resource Manager templates, enabling consistent deployments across AWS and Azure environments
    
    \item Built automated security scanning into IaC pipelines using Python scripts, identifying and preventing 95\% of common misconfigurations before deployment
    
    \item Created documentation and best practices for cloud infrastructure management, adopted by 3 development teams for their projects
    
\end{highlights}
\end{onecolentry}

\vspace{0.2cm}


\section{Technical Skills}
\begin{minipage}[t]{\linewidth}
\begin{itemize}[nosep,after=\strut, leftmargin=1em, itemsep=2pt]

\item Programming Languages: Python, Java, Golang, JavaScript, SQL

\item Cloud \& Infrastructure: AWS, Azure, Docker, Kubernetes, EKS, ECS, Serverless, IaaS, Storage, Security, Identity

\item Infrastructure as Code: Terraform, CloudFormation, Azure Resource Manager, Pulumi, OpenTofu

\item DevOps \& Tools: CI/CD, Jenkins, GitHub Actions, Git, Cloud Monitoring tools, Prometheus, Grafana

\item Architecture \& Databases: Microservices, Containers, Container Orchestration, RDBMS, NoSQL, REST APIs, Distributed Systems

\item Emerging Technologies: LLM, Agentic AI, A2A, MCP, Rapid Prototyping

\end{itemize}
\end{minipage}

\vspace{0.2cm}

\section{Education}
\begin{twocolentry}{Aug 2023 -- May 2025}
    \textbf{The University of Texas at Arlington,} MS in Data Science
\end{twocolentry}
\begin{onecolentry}
\begin{highlights}
    \item \textbf{GPA:} 3.8/4.0
    \item \textbf{Relevant Coursework:} Distributed Systems, Cloud Computing, Software Engineering, Data Structures, Algorithms, Machine Learning
\end{highlights}
\end{onecolentry}

\end{document}