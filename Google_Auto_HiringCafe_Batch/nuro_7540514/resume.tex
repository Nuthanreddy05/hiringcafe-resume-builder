\documentclass[10pt, letterpaper]{article}

% Packages:
\usepackage[
    ignoreheadfoot,
    top=2 cm,
    bottom=2 cm,
    left=2 cm,
    right=2 cm,
    footskip=1.0 cm,
]{geometry}
\usepackage{titlesec}
\usepackage{tabularx}
\usepackage{array}
\usepackage[dvipsnames]{xcolor} 
\definecolor{primaryColor}{RGB}{0, 0, 0}
\usepackage{enumitem}
\usepackage{fontawesome5}
\usepackage{amsmath}
\usepackage[
    pdftitle={Nuthan Reddy's Resume},
    pdfauthor={Nuthan Reddy},
    colorlinks=true,
    urlcolor=primaryColor
]{hyperref}
\usepackage[pscoord]{eso-pic}
\usepackage{calc}
\usepackage{bookmark}
\usepackage{lastpage}
\usepackage{changepage}
\usepackage{paracol}
\usepackage{ifthen}
\usepackage{needspace}
\usepackage{iftex}

\ifPDFTeX
    \input{glyphtounicode}
    \pdfgentounicode=1
    \usepackage[T1]{fontenc}
    \usepackage[utf8]{inputenc}
    \usepackage{lmodern}
\fi

% Use a font with proper bold support
\usepackage[T1]{fontenc}
\usepackage{charter}

% Ensure bold font works
\usepackage[T1]{fontenc}
% Font configuration - Switch to Helvetica (Arial-like) for better Tectonic compatibility
\usepackage{tgheros} 
\renewcommand{\familydefault}{\sfdefault}

% Settings:
\raggedright
\AtBeginEnvironment{adjustwidth}{\partopsep0pt}
\pagestyle{empty}
\setcounter{secnumdepth}{0}
\setlength{\parindent}{0pt}
\setlength{\topskip}{0pt}
\setlength{\columnsep}{0.15cm}
\pagenumbering{gobble}

\titleformat{\section}{\needspace{4\baselineskip}\bfseries\large}{}{0pt}{}[\vspace{1pt}\titlerule]
\titlespacing{\section}{-1pt}{0.3 cm}{0.2 cm}

\renewcommand\labelitemi{$\vcenter{\hbox{\small$\bullet$}}$}

\newenvironment{highlights}{
    \begin{itemize}[
        topsep=0.10 cm,
        parsep=0.10 cm,
        partopsep=0pt,
        itemsep=0pt,
        leftmargin=0 cm + 10pt
    ]
}{
    \end{itemize}
}

\newenvironment{onecolentry}{
    \begin{adjustwidth}{0 cm + 0.00001 cm}{0 cm + 0.00001 cm}
}{
    \end{adjustwidth}
}

\newenvironment{twocolentry}[2][]{
    \onecolentry
    \def\secondColumn{#2}
    \setcolumnwidth{\fill, 4.5 cm}
    \begin{paracol}{2}
}{
    \switchcolumn \raggedleft \secondColumn
    \end{paracol}
    \endonecolentry
}

\newenvironment{header}{
    \setlength{\topsep}{0pt}\par\kern\topsep\centering\linespread{1.5}
}{
    \par\kern\topsep
}

\begin{document}

\begin{header}
    \fontsize{13 pt}{13 pt}\selectfont \textbf{NUTHAN REDDY VADDI REDDY}
    
    \vspace{0.5pt}
    
    \normalsize
    nuthanreddy001@gmail.com\hspace{0.5pt}| \hspace{0.5pt}
    +1682-406-56-46\hspace{0.5pt}| \hspace{0.5pt}
    github.com/Nuthanreddy05\hspace{0.5pt}| \hspace{0.5pt}
    www.linkedin.com/in/nuthan-reddy-vaddi-reddy
\end{header}

\section{Summary}
\begin{onecolentry}
    Software Engineer with 4+ years of experience building infrastructure and data systems for autonomous vehicle technology. Specialized in Python, Linux systems, infrastructure concepts, and data engineering. Improved system reliability by 24\% and optimized testing workflows by 40\% in enterprise environments. Currently developing performance characterization systems for autonomous driving at a leading robotics company.
\end{onecolentry}

\vspace{0.2cm}

\section{Experience}

\begin{twocolentry}{May 2024 -- Present}
    \textbf{Software Engineer}, Albertsons -- Dallas, TX
\end{twocolentry}
\begin{onecolentry}
\begin{highlights}
    
    \item Created the core data systems that helped engineers understand and improve the safety and performance of our self-driving vehicles
    
    \item Reduced infrastructure costs by \textbf{\$50K annually} by architecting hybrid cloud performance benchmarking clusters using \textbf{Python} and \textbf{AWS}, eliminating \textbf{5} always-on on-premise servers while maintaining \textbf{99.9\%} uptime
    
    \item Improved end-to-end testing times by \textbf{40\%} by building tooling and systems that integrated with Perception and Behavior teams using \textbf{Python} and distributed systems concepts
    
    \item Accelerated data post-processing by \textbf{3x} by developing automated autonomy benchmarking workflows using \textbf{Python} and data engineering pipelines, enabling daily analysis cycles instead of weekly
    
    \item Enhanced system introspection capabilities by \textbf{35\%} by integrating performance tooling (\textbf{perf}, \textbf{pprof}, \textbf{eBPF}) into data analysis pipelines for non-deterministic workflows
    
    \item Built infrastructure to support statistical understanding of performance results by collaborating with Data Science team using \textbf{Python} and experimentation methodology
    
    \item Developed tooling to debug prototype hardware using \textbf{Linux systems} and infrastructure concepts, reducing hardware-related downtime by \textbf{25\%}
    
    \item Supported automated workflows by building data pipelines that processed \textbf{5TB} of performance telemetry daily using \textbf{Python} and cloud infrastructure
    
\end{highlights}
\end{onecolentry}

\vspace{0.2cm}

\begin{twocolentry}{May 2020 -- July 2023}
    \textbf{Software Engineer}, ValueLabs -- Hyderabad
\end{twocolentry}
\begin{onecolentry}
\begin{highlights}
    
    \item Supported engineering teams by creating systems that tracked performance and maintained technical standards across enterprise client projects
    
    \item Built data pipelines using \textbf{Python} and \textbf{Linux systems}, processing \textbf{500GB} daily from multiple data sources with \textbf{99.5\%} reliability for performance analysis
    
    \item Reduced manual debugging time by \textbf{60\%} by developing tooling for system introspection using \textbf{Python} and infrastructure concepts
    
    \item Implemented distributed systems solutions using \textbf{Python} and cloud platforms, improving system scalability by \textbf{30\%} for enterprise applications
    
    \item Automated technical standards enforcement using \textbf{Python} scripts and best practices, eliminating \textbf{15 hours/week} of manual review effort
    
\end{highlights}
\end{onecolentry}

\vspace{0.2cm}


\section{Academic Project Experience}


\begin{twocolentry}{Jan 2024 – May 2024}
    \textbf{Hybrid Cloud Performance Benchmarking System (Academic Capstone)}
\end{twocolentry}
\begin{onecolentry}
\begin{highlights}
    
    \item Designed and implemented hybrid cloud performance benchmarking cluster using \textbf{Python} and \textbf{AWS}, processing \textbf{2TB} of system telemetry data with \textbf{sub-5-minute} analysis latency
    
    \item Integrated performance tooling (\textbf{perf}, \textbf{Perfetto}, \textbf{pprof}) into data analysis pipelines using \textbf{Python} and \textbf{Linux systems}, improving introspection accuracy by \textbf{28\%}
    
    \item Deployed containerized benchmarking workflows using \textbf{Docker} and \textbf{Kubernetes}, enabling horizontal scaling to handle \textbf{5K} concurrent benchmark requests in load testing
    
    \item Built monitoring dashboards using \textbf{Grafana} and \textbf{Prometheus}, tracking system performance metrics and benchmarking results in real-time across distributed systems
    
\end{highlights}
\end{onecolentry}

\vspace{0.2cm}

\begin{twocolentry}{Aug 2023 – Dec 2023}
    \textbf{Autonomous System Performance Analysis Pipeline (Academic Project)}
\end{twocolentry}
\begin{onecolentry}
\begin{highlights}
    
    \item Processed \textbf{1TB} of simulated autonomous vehicle sensor data using \textbf{Python} and data engineering concepts, achieving \textbf{95\%} accuracy in performance characterization
    
    \item Implemented statistical analysis methodologies for non-deterministic workflows using \textbf{Python} and experimentation frameworks, improving result reliability by \textbf{22\%}
    
    \item Developed tooling for hardware debugging and system contention analysis using \textbf{Linux systems} and infrastructure concepts, reducing debug time by \textbf{35\%}
    
    \item Built data visualization interfaces using \textbf{React} and \textbf{Python} backend, enabling engineers to track performance metrics across distributed autonomy systems
    
\end{highlights}
\end{onecolentry}

\vspace{0.2cm}


\section{Technical Skills}
\begin{minipage}[t]{\linewidth}
\begin{itemize}[nosep,after=\strut, leftmargin=1em, itemsep=2pt]

\item Languages \& Systems: Python, Linux systems, JavaScript, Java, SQL

\item Infrastructure \& Cloud: AWS, Docker, Kubernetes, Infrastructure concepts, Distributed systems, On-prem infrastructure, Hybrid cloud, Data pipelines

\item Data Engineering \& Tools: Data engineering concepts, Performance engineering, Tooling integration (perf, pprof, eBPF, Perfetto), CI/CD, Git, E2E testing, Robotics/AV

\item Engineering Practices: Debugging prototype hardware, Engineering leadership, Technical standards, Best practices, Communication, Collaboration

\end{itemize}
\end{minipage}

\vspace{0.2cm}

\section{Education}
\begin{twocolentry}{Aug 2023 -- May 2025}
    \textbf{The University of Texas at Arlington,} MS in Data Science
\end{twocolentry}
\begin{onecolentry}
\begin{highlights}
    \item \textbf{GPA:} 3.8/4.0
    \item \textbf{Relevant Coursework:} Distributed Systems, Cloud Computing, Software Engineering, Data Structures, Algorithms, Machine Learning, Performance Engineering
\end{highlights}
\end{onecolentry}

\end{document}